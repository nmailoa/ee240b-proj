\documentclass[conference]{IEEEtran}
\ifCLASSINFOpdf
   \usepackage[pdftex]{graphicx}
\else
\fi
\usepackage{amsmath}
\usepackage{stfloats}
\usepackage{verbatim}
\hyphenation{op-tical net-works semi-conduc-tor}


\begin{document}
\title{Design of Oversampled ADC Integrators \\ \Large EE240B Project Report}
\author{\IEEEauthorblockN{Emily Naviasky}
\IEEEauthorblockA{Department of EECS, UC Berkeley \\enaviasky@berkeley.edu}
\and
\IEEEauthorblockN{Nathaniel Mailoa}
\IEEEauthorblockA{Department of EECS, UC Berkeley \\nmailoa@eecs.berkeley.edu}
}
\maketitle

\section{Introduction}

We will propose a design for the integrator stages of an oversampled ADC. The design is driven by some specifications that include input-referred electronic noise, settling time as well as settling accuracy. We have chosen to use the same integrator architecture for both stages but tune the component values for each stage.


\section{Integrator Implementation}



\section{Design Alternatives}



\section{Circuit Implementation}



\section{Design Verification}



\section{Final Specification}

\begin{center}
\begin{tabular}{|c|c|} 
\hline
Gain & 0.5 \\
\hline
Input Referred Noise & ?? $\mu$ $V_{RMS}$ \\
\hline
Settling & 0.1\% in ?? ns \\
\hline
Sampling Freq & 250M Hz \\
\hline
Total Power & ?? mW \\
\hline
\end{tabular}
\end{center}



\section{Design Critique}

flicker? - chopper stabilization








\begin{comment}


\section{Design}
For this project the important specifications are:

\begin{center}
\begin{tabular}{|c|c|} 
\hline
Gain (Stage 1) & 0.5 \\
\hline
Input Referred Noise & $10\mu$ $V_{RMS}$ \\
\hline
Settling & 0.1\% in 1.8ns \\
\hline
Sampling Freq & 250M Hz \\
\hline
\end{tabular}
\end{center}

We divided the settling error requirement evenly between static and dynamic error, so $\epsilon_d = 0.05\%$. Furthermore, we decided to budget noise based on the noise factor calculations from the previous section. We see that the noise $\phi_1$ is the most important contribution to the noise. $\phi_2$ will be divided by a factor of 1000, therefore budget nearly all of the noise requirements for noise from $N_{11}$. The spec is for differential noise, so calculating single ended, we need to hit $5\mu$ $V_{RMS}$. We divide that further up, so that there is $3\mu$ $V_{RMS}$ for one input of the first integrator, and $1\mu$ $V_{RMS}$ for one input of the second stage and output of the second stage, taking advantage of the fact that the second integrator can be designed with much looser noise requirements.

From these specifications, we see that the small noise and rapid settling time will require careful design to meet both simultaneously. We began by considering a basic switched capacitor integrator in fig. \ref{int1} to get an idea of the what values were needed to meet $3\mu$ $V_{RMS}$ per input. We can use a design process similar to that of a sample and hold circuit, with only minor differences due to different switches.

\begin{figure}[h]
\centering
\includegraphics[width=0.4\linewidth]{illustrator/integrator1}
\caption{Basic switched capacitor gain stage}
\label{int1}
\end{figure}

The necessary equations are below. \newline
Gain:
$$\frac{V_{out}}{V_{in}}=\frac{C_S}{C_F}$$
Settling Time:
$$t_s = -\tau ln \left(\epsilon_d \left(1-\beta \frac{C_F}{C_F-C_L}\right)\right)$$
Noise:
$$N_{\phi 1} = \frac{kT}{C_S + C_L}$$
$$N_{\phi 2} = \frac{\alpha}{\beta}\frac{kT}{C_{L,tot}}$$
Where in these equations $\alpha$ is the noise factor of the OTA - we used $\alpha = 2$ for the design - and $\beta$ is the feedback factor. $$\beta = \frac{C_F}{C_S + C_F + C_L}$$ 
In addition, $C_{L,tot}$ is the effective capacitive load.$$C_{L,tot}=C_L+(1-\beta)C_F$$
And $\tau$ is the time constant of the system. $$\tau = \frac{C_{L,tot}}{\beta G_m}$$

\subsection{First Integrator Stage}

We begin by choosing $C_S$ from the noise spec and the equation for $\phi_1$. Then $C_F$ is easily obtained from the required gain. We want to meet $3\mu$ $V_{RMS}$ per input, but the BW that we care about noise is only 100-500kHz and our equation for $N_{\phi 1}$ is over the entire spectrum [1]. Therefore, we need to multiply the noise spec by the Oversampling Ratio, which is $\frac{f_S}{2BW}$. The noise that we want at the input therefore is $N_{11}=250 \cdot 9\cdot 10^{-12} V/\sqrt{Hz}$.
%IMPT VALS
$$C_S = 2pF$$ 
$$C_F = 4pF$$
These are somewhat large capacitor values to drive. We next calculate the $G_m$ of the OTA necessary to meet the settling time with those capacitors from $\tau$. We get $G_m = 9mS$, which is a little large, but should be doable with large width MOSFETs.
We began adding parasitics to the transconductor. Parasitic capacitors are from $C_{GS}=\frac{g_m}{\omega_T}$ and we made a first optimistic guess at a conservative $r_o=10k$ and found that in simulation we needed significantly larger $G_m$. A small $G_m$ would result in a small loop gain due to the non-infinite $r_o$, which creates a significant static settling error.
We decided that $G_m$ was getting a little large and decided to check a cascaded setup shown in fig. \ref{int2}. 

\begin{figure}[h]
\centering
\includegraphics[width=0.7\linewidth]{illustrator/integrator2}
\caption{Cascaded amplifiers with Miller compensation}
\label{int2}
\end{figure}
 
The relevant equations for this part change only a little:
$$\tau = \frac{C_C}{\beta g_{m1}}$$
$$N_{\Phi 2} = \frac{\alpha_1}{\beta} \frac{kT}{C_C} \left( 1+\beta \frac{\alpha_2}{\alpha_1} \frac{C_C}{C_{Ltot}} \right)$$

However, most of the equations for poles and zeros are very approximated, and they do not take into account that the cascaded amplifiers are more stable when $g_{m1}$ is larger than $g_{m2}$. We also had to use a Miller capacitance for pole splitting.

We had to iteratively find a stable value of $g_m$s that met timing, and eventually settled on $g_{m1}=20mS$, $g_{m2}=4mS$.

\begin{figure}[h]
\centering
\includegraphics[width=\linewidth]{img/cascaded-tran}
\caption{Transient response of 2-stage amplifier with Miller compensation}
\label{cascaded-tran}
\end{figure}

These value of $g_m$ are not significantly better than the single stage transconductor and the noise of a single stage is better, so we decided that a very large diff pair would be better than cascading.

We re-examined the parasitic resistance and decided that increasing the $r_o$ was the next best option, so that we needed the first stage OTA to be a single stage diff pair with cascoding, and approximated the new $r_o$ as a conservative $100k\Omega$.

We ran a final transient sim to make sure that the settling time of the output of the first stage was 1.8ns to 0.1\%. 

\begin{figure}[h]
\centering
\includegraphics[width=\linewidth]{img/stage1-tran}
\caption{Transient response of first integrator with single-stage amplifier}
\label{stage1-tran}
\end{figure}

The noise simulation was a performed with a noise current source from a resistor of value $\frac{\alpha}{G_m}$ inside of the OTA and $10\Omega$ ON resistances in series with all of the switches. However, PSS and PNOISE are not really meant for integrators so we exported the noise density at the output of the integrator to Matlab, then multiplied by the transfer function from the noise source $N_{12}$ to the input and then integrated over the BW.

\begin{figure}[h]
\centering
\includegraphics[width=\linewidth]{img/noise-stage1}
\caption{Input-referred noise density from the first integrator stage}
\label{noise-stage1}
\end{figure}

The integrated noise is $2.245\mu V_{RMS}$ which gives us a lot of room to change $C_s$ if we need it.

\begin{center}
\begin{tabular}{|c|c|} 
\hline
$C_S$ & 2pF \\
\hline
$C_F$ & 4pF \\
\hline
$G_M$ & 19mS \\
\hline
$R_{o,min}$ & 100k$\Omega$ \\
\hline
\end{tabular}
\end{center}

In this stage, the most important design factor is meeting the noise requirement. This means that if noise is a problem later in the problem we can increase $C_S$. However, since we are very cleanly meeting the noise spec, we can trade some of it for lower $G_m$ in the gain stage. The large capacitors in the feedback loop relative to the load capacitance of the next stage mean we don't need quite as good of output resistance to keep static error within spec.

\subsection{Second Integrator Stage}

For second stage, we are meeting a noise spec which is 1000 times more lenient. Thus, we can choose much smaller feedback capacitance and $G_m$ of the OTA can be much more lenient. Using the same design procedure as before we obtain the values for the second integrator, also with a gain of 0.5. A larger gain would marginally help with the noise requirements, but a smaller gain would result in a higher $\beta$, hence higher loop gain. This is required once we have some load resistance at the output to meet the static settling error with a reasonable $G_m$.

\begin{center}
\begin{tabular}{|c|c|} 
\hline
$C_S$ & 30fF \\
\hline
$C_F$ & 15fF \\
\hline
$G_M$ & 5mS \\
\hline
$R_{o,min}$ & 1M$\Omega$ \\
\hline
\end{tabular}
\end{center}


We did find, that because the feedback capacitors are not on the same order as the load cap, the static error is a much bigger concern, so we need to meet a larger output resistance of the OTA in order to keep the $g_m$ build-able with one stage.

\begin{figure}[h]
\centering
\includegraphics[width=\linewidth]{img/stage2-tran}
\caption{Transient response of second integrator with single-stage amplifier}
\label{stage2-tran}
\end{figure}

\begin{figure}[h]
\centering
\includegraphics[width=\linewidth]{img/noise-stage2}
\caption{Input-referred noise density from the second integrator stage}
\label{noise-stage2}
\end{figure}


\section{Implementation}

\subsection{OTAs}
The above simulations confirmed that we needed reasonably large $G_m$ OTAs, and that output resistance is more important for the second stage than the first stage.

\begin{figure}[h]
\centering
\includegraphics[width=0.4\linewidth]{illustrator/cascoded-diffpair}
\caption{Cascoded fully-differential pair}
\label{cascoded-diffpair}
\end{figure}

For the first stage we want a differential pair with reasonably large output resistance. We propose the cascoded fully-differential pair in fig. \ref{cascoded-diffpair}.
$M_1$ and $M_2$ completely determine the $G_m$ of the OTA and will need to be sized accordingly. The large width of $M_1$ and $M_2$ necessitates using cascoding at least on the input caps, but careful sizing of $M_5$ and $M_6$ should meet the required output resistance of the first stage without having to use a full telescopic design which will limit the output swing a lot.

For the second stage, however, we need better output resistance, in which case we need a telescopic differential pair, but we should not need gain boosting. The output voltage swing is not defined in the spec, so cascoded devices is feasible.

\subsection{Switches}
The resistance of the switches is somewhat of a concern, especially for the sampling switch. The time constant of this switch with $C_S$ needs to be substantially low such that the sampling capacitor has time to charge sufficiently, and that the switch can drive over the large input range. Therefore, the first switch, at least, needs to be a reasonably wide transmission gate.

The problem with large switches is that they have a lot of charge injection. To avoid that, we are making sure that we use early switches to disconnect the sampling cap and to implement bottom plate sampling.

\begin{figure}[h]
\centering
\includegraphics[width=0.8\linewidth]{illustrator/integrator3}
\caption{Integrator stage with added features}
\label{integrator3}
\end{figure}


\subsection{Integrators}
At the integrator level, we have been dealing only with the specs for this project. The project does not specify a CMRR or amount of variation that we have to be able to deal with in fabrication. However, these are things that have to be dealt with.

We propose the integrator implementation in fig.\ref{integrator3} to deal with problems outside of simulation.

The switch between the inputs to the OTA marked with the red box is to improve the CMRR by canceling the common mode offset [2]. The effect of OTA offset from fabrication mismatch is negligible because Delta-Sigma modulators are relatively insensitive to offset [3].


\end{comment}


%\subsection{Subsection Heading Here}
%Subsection text here.


%\subsubsection{Subsubsection Heading Here}
%Subsubsection text here.


% An example of a floating figure using the graphicx package.
% Note that \label must occur AFTER (or within) \caption.
% For figures, \caption should occur after the \includegraphics.
% Note that IEEEtran v1.7 and later has special internal code that
% is designed to preserve the operation of \label within \caption
% even when the captionsoff option is in effect. However, because
% of issues like this, it may be the safest practice to put all your
% \label just after \caption rather than within \caption{}.
%
% Reminder: the "draftcls" or "draftclsnofoot", not "draft", class
% option should be used if it is desired that the figures are to be
% displayed while in draft mode.
%
%\begin{figure}[!t]
%\centering
%\includegraphics[width=2.5in]{myfigure}
% where an .eps filename suffix will be assumed under latex, 
% and a .pdf suffix will be assumed for pdflatex; or what has been declared
% via \DeclareGraphicsExtensions.
%\caption{Simulation results for the network.}
%\label{fig_sim}
%\end{figure}

% Note that the IEEE typically puts floats only at the top, even when this
% results in a large percentage of a column being occupied by floats.


% An example of a double column floating figure using two subfigures.
% (The subfig.sty package must be loaded for this to work.)
% The subfigure \label commands are set within each subfloat command,
% and the \label for the overall figure must come after \caption.
% \hfil is used as a separator to get equal spacing.
% Watch out that the combined width of all the subfigures on a 
% line do not exceed the text width or a line break will occur.
%
%\begin{figure*}[!t]
%\centering
%\subfloat[Case I]{\includegraphics[width=2.5in]{box}%
%\label{fig_first_case}}
%\hfil
%\subfloat[Case II]{\includegraphics[width=2.5in]{box}%
%\label{fig_second_case}}
%\caption{Simulation results for the network.}
%\label{fig_sim}
%\end{figure*}
%
% Note that often IEEE papers with subfigures do not employ subfigure
% captions (using the optional argument to \subfloat[]), but instead will
% reference/describe all of them (a), (b), etc., within the main caption.
% Be aware that for subfig.sty to generate the (a), (b), etc., subfigure
% labels, the optional argument to \subfloat must be present. If a
% subcaption is not desired, just leave its contents blank,
% e.g., \subfloat[].


% An example of a floating table. Note that, for IEEE style tables, the
% \caption command should come BEFORE the table and, given that table
% captions serve much like titles, are usually capitalized except for words
% such as a, an, and, as, at, but, by, for, in, nor, of, on, or, the, to
% and up, which are usually not capitalized unless they are the first or
% last word of the caption. Table text will default to \footnotesize as
% the IEEE normally uses this smaller font for tables.
% The \label must come after \caption as always.
%
%\begin{table}[!t]
%% increase table row spacing, adjust to taste
%\renewcommand{\arraystretch}{1.3}
% if using array.sty, it might be a good idea to tweak the value of
% \extrarowheight as needed to properly center the text within the cells
%\caption{An Example of a Table}
%\label{table_example}
%\centering
%% Some packages, such as MDW tools, offer better commands for making tables
%% than the plain LaTeX2e tabular which is used here.
%\begin{tabular}{|c||c|}
%\hline
%One & Two\\
%\hline
%Three & Four\\
%\hline
%\end{tabular}
%\end{table}


% Note that the IEEE does not put floats in the very first column
% - or typically anywhere on the first page for that matter. Also,
% in-text middle ("here") positioning is typically not used, but it
% is allowed and encouraged for Computer Society conferences (but
% not Computer Society journals). Most IEEE journals/conferences use
% top floats exclusively. 
% Note that, LaTeX2e, unlike IEEE journals/conferences, places
% footnotes above bottom floats. This can be corrected via the
% \fnbelowfloat command of the stfloats package.




% conference papers do not normally have an appendix


% use section* for acknowledgment
%\section*{Acknowledgment}


%The authors would like to thank...





% trigger a \newpage just before the given reference
% number - used to balance the columns on the last page
% adjust value as needed - may need to be readjusted if
% the document is modified later
%\IEEEtriggeratref{8}
% The "triggered" command can be changed if desired:
%\IEEEtriggercmd{\enlargethispage{-5in}}

% references section

% can use a bibliography generated by BibTeX as a .bbl file
% BibTeX documentation can be easily obtained at:
% http://mirror.ctan.org/biblio/bibtex/contrib/doc/
% The IEEEtran BibTeX style support page is at:
% http://www.michaelshell.org/tex/ieeetran/bibtex/
%\bibliographystyle{IEEEtran}
% argument is your BibTeX string definitions and bibliography database(s)
%\bibliography{IEEEabrv,../bib/paper}
%
% <OR> manually copy in the resultant .bbl file
% set second argument of \begin to the number of references
% (used to reserve space for the reference number labels box)
\begin{thebibliography}{1}

\bibitem{}
R. Schreier, \emph{Design-Oriented Estimation of Thermal Noise in Switched-Capacitor Circuits},  IEEE Transactions on Circuits and Systems—I: Regular Papers, vol.52, no.11, 2005.

\bibitem{}
S. Lewis, P. Gray. \emph{A Pipelined 5-Msample/s 9-bit Analog-to-Digital Converter}, IEEE Journal of Solid-State Circuits, vol.sc-22 ,no.6, 1987.

\bibitem{}
B. Bernhard, \emph{The Design of Sigma-Delta Modulation Analog-to-Digita1 Converters}, IEEE Journal of Solid-State Circuits, vol.23 ,no.6, 1988.
\end{thebibliography}




% that's all folks
\end{document}
